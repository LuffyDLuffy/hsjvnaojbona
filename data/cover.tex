\bnusetup{
  %******************************
  % 注意:
  %   1. 配置里面不要出现空行
  %   2. 不需要的配置信息可以删除
  %******************************
  %
  %=====
  % 中文信息
  %=========
  ctitle={等温捷径和绝热捷径构成的类卡诺循环热机的功率与效率},
  cdegree={理学学士},
  cdepartment={物理学系},
  cmajor={物理学},
  cauthor={龚政楠},
  csupervisor={涂展春~教授},
  cassosupervisor={北京师范大学~物理学系}, % 校内
  % 日期自动使用当前时间,若需指定按如下方式修改:
  %=========
  % 英文信息
  %=========
  etitle={Power and efficiency of a Carnot-like cycle heat engine consisting of the isothermal shortcuts and the adiabatic shortcuts},
  edegree={Bacholar of Science},
  emajor={Physics},
  eauthor={ZN.G},
  esupervisor={XXX},
  eassosupervisor={XXXX},
  % 日期自动生成,若需指定按如下方式修改:
  % edate={December, 2005}
  %
  % 关键词用“英文逗号”分割
  ckeywords={随机热机,绝热捷径,等温捷径,最大功率时的效率},
  ekeywords={stochastic heate engine, adiabatic shortcuts, isothermal shortcuts, efficiency at maximum power}
}

% 定义中英文摘要和关键字
\begin{cabstract}
热机的效率问题一向深受研究者的关注。虽然卡诺已经得到热机效率的极限,但鉴于卡诺循环热机要求循环过程是可逆的,而这会导致卡诺热机的输出功率为0。人们开始思考如何加速可逆过程以获得有限的功率,这导致了有限时间热力学的诞生。在如何平衡效率与功率的问题上,Curzon 和 Ahlborn为后来的研究者树立了范式。1975年,他们构建了可在有限时间内运行的CA内可逆热机,并且计算出了该热机在最大功率时的效率。在这以后,研究者又构建了形形色色的可在有限时间内运行的热机,并且计算了它们在最大功率时的效率。其中,T. Schmiedl 和 U. Seifert 提出的微观随机热机模型通过外势场驱动布朗粒子对外做功,以随机热力学为基础的随机热机随之出现在人们眼前。

随机热力学发端于人们对小系统的热力学性质的研究,这种纳米尺度的小系统包含的粒子数要远小于$N_{\M{A}}$。所以小系统的涨落效应和脱离平衡态的的程
度都很大,传统热力学对于处理这样的系统力不从心。Ken Sekimoto和Udo Seifert等人将经典热力学中的概念如功、热、熵等推广得到了单个轨道的情形,这是我们处理随机热机的功率与效率问题的基础。

在有限时间热力学中的一个关键问题就是如何在有限时间内实现平衡态的转化。Martínez针对谐振子场中的布朗粒子,通过对劲度系数进行调节,使系统比自然驰豫过程快一百倍达到了平衡态。而Le Cunuder 等人也实现了微观谐振子平衡态的快速转化……在这许许多多的实现平衡态转化的策略中,绝热捷径和等温捷径的策略占有一席之地。通过为原系统引入辅助哈密顿量和辅助势,两者通过将系统的状态保持在原哈密顿量的平衡态下,进而实现了平衡态的转化。这也是捷径一词的由来。

于是我们利用绝热捷径和等温捷径构建了一个类卡诺热机模型,再结合随机热力学的知识,计算了热机的效率与功率。通过借鉴T. Schmiedl 和 U. Seifert等人处理微观随机热机的方法,利用求多元函数极值和泛函极值的办法,在过阻尼和欠阻尼的情形下,得到了构建的热机在最大功率时的效率。
\end{cabstract}

% 如果习惯关键字跟在摘要文字后面,可以用直接命令来设置,如下:
% \ckeywords{\TeX, \LaTeX, CJK, 模板, 论文}

\begin{eabstract}
The efficiency of heat engines has always been of great interest to researchers. Although Carnot had already obtained the limit of heat engine efficiency, since the Carnot cycle heat engine required the cycle process to be reversible, which resulted in zero output power of the Carnot heat engine, people started to think about how to accelerate the reversible process to obtain finite power, which led to the birth of Finite Time Thermodynamics(FTT). Curzon and Ahlborn set the paradigm for later researchers on how to balance efficiency and power. In 1975, they constructed a CA internal reversible heat engine that could operate in finite time and calculated the efficiency of the heat engine at maximum power. Since then, researchers have constructed various types of heat engines that can operate in finite time and calculated their efficiency at maximum power. Among them, the microscopic stochastic heat engine model proposed by T. Schmiedl and U. Seifert, which is based on stochastic thermodynamics, emerged by driving Brownian particles to do external work through an external potential field. And the stochastic heat engine based on stochastic thermodynamics appears in people’s eyes.

Stochastic thermodynamics began with the study of the thermodynamic properties of small systems, which contain far fewer particles than $N_{\M{A}}$ at the nanoscale. Seifert \textit{et al}. extended the concepts of classical thermodynamics, such as work, heat, and entropy, to the case of a single orbit, which is the basis for our treatment of the power and efficiency problems of stochastic heat engines. 

A key problem in finite time thermodynamics is how to achieve the equilibrium transition in finite time, and Martínez, for Brownian particles in a harmonic oscillator field, adjusted the stiffness coefficient so that the system reaches the equilibrium state 100 times faster than the natural chirality process. The fast transformation of microscopic resonator equilibria was also achieved by Le Cunuder \textit{et al}. ...... Among the many strategies to achieve equilibrium transformation, adiabatic shortcuts and isothermal shortcuts have a place. By introducing auxiliary Hamiltonian and auxiliary potential to the original system, both of them achieve the equilibrium transition by keeping the system in the equilibrium state of the original Hamiltonian. This is where the term shortcut comes from. 

Thus, a Carnot-like heat engine model is constructed using adiabatic shortcuts and isothermal shortcuts, and the efficiency and power of the heat engine are calculated by combining the knowledge of stochastic thermodynamics. The efficiency of the constructed heat engine at maximum power is obtained by using the multivariate function extremum and the funtional extremum in the overdamped and underdamped cases, drawing on the approach of T. Schmiedl and U. Seifert \textit{et al}. for microscopic stochastic heat engines.
\end{eabstract}

% \ekeywords{\TeX, \LaTeX, CJK, template, thesis}