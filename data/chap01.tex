\chapter{绪论:从经典热力学到随机热力学}
\label{c1}
热机的发明和使用对人类的生产生活产生了重大的影响,第一次工业革命和第二次工业革命都和热机的发展有紧密的关系。 一直以来,特别是近些年来,由于能源短缺的问题愈发突出,研究热机的效率问题也一直吸引着一大批研究者。

\section{引言:热机的最大功率与效率问题}
\qquad 早在1824,卡诺(Sadi Carnot)就指出(基于热质说)\cite{2005}:工作在相同高温热源$T_{\M{h}}$和低温热源$T_{\M{c}}$的所有热机中,以可逆热机的效率最高,而且这些可逆热机的效率都相同,为$\eta _{\text{c}}=1-\frac{T_{\M{c}}}{T_{\M{h}}}$ 。这个问题似乎就这么解决了,然而必须注意的是卡诺热机要求循环过程是可逆的。也就是说,要想达到卡诺热机的效率,热机一个循环的工作时间应当是无限长,这样的热机功率为0。

我们当然不可能接受功率为0的热机,所以研究者开始关注热机在某一功率下(特别是最大功率)的最大效率,而这个问题促进了有限时间热力学的诞生。

1975年,Curzon和Ahlborn构建了了内可逆热机\cite{Curzon1975},该热机可在有限时间内完成一个循环。并且得到了该热机在最大功率下的效率为$\eta _{\mathrm{CA}}=1-\sqrt{T_{\mathrm{c}}/T_{\mathrm{h}}}=1-\sqrt{1-\eta _{\mathrm{c}}}$ ,其中$\eta _{\mathrm{c}}$为卡诺效率。

2008年,T. Schmiedl 和 U. Seifert提出了微观随机热机模型\cite{Schmiedl2008},该热机模型利用依赖于时间的谐振子势场驱动布朗粒子做功,并得到了其在最大功率下的效率:$\eta_{\M{TU}}={ \eta_{0}}/{(2-\eta_{0}/2)}$

同年,涂展春推导出费曼棘轮热机\cite{Tu2008}在最大功率下的效率$\eta _{\text{F}}=\frac{\eta _{\text{C}}^{2}}{\eta _{\text{C}}-\left( 1-\eta _{\text{C}} \right) \ln \left( 1-\eta _{\text{C}} \right)}$。 \cite{Tu2020}

……

同时涂展春\cite{Tu2008}在小温差条件下,即$\eta _{\text{c}}\rightarrow 0$。 将上述热机的效率按照 展开. 发现这些效率到二阶项都相同的,它们都普遍地满足$\eta _{\text{U}}=\frac{\eta _{\text{c}}}{2}+\frac{\eta _{\text{c}}^{2}}{8}+O\left( \eta _{\text{c}}^{3} \right)$。 这个规律在其他热机中也得到了验证,但也有一些热机不满足这个关系. 这引发了研究者对相关问题的思考,涂展春在紧耦合热机的范畴内对其中的原因进行了解释.\cite{Tu2020}

在这些形形色色的有限时间热机中,由Seifert等人提出的微观随机热机\cite{Schmiedl2008}占据一席之地. 随机热机通常是利用外势驱动粒子对外做功. 其中的具体的实现过程可以有很大的不同,对应着不同类型的随机热机。2014年,Campo等人利用绝热捷径构建了奥托热机\cite{DelCampo2014}。 而后,Deng等人发现绝热捷径能够提高奥托热机的效率,不论是经典的还是量子的情形下。\cite{Deng2013} 2014年,涂展春在Schmiedl和Seifert工作\cite{Schmiedl2008}的基础上,考虑了他们在过阻尼情况下忽略的粒子的惯性的影响,构建了一种类卡诺热机\cite{Tu2013},并且惊讶地发现这种随机热机的效率等于Curzon和Ahlborn构建的内可逆热机\cite{Curzon1975}的效率。在涂展春构建的热机中,布朗粒子在与时间相关的谐振子$U=\lambda ^2\left( t \right) q^2/2$的控制下,经历了类卡诺循环. 其中与卡诺循环中绝热过程对应的就是前面提到的绝热捷径,但与卡诺循环中等温过程对应的“等温过程”,却只是粒子与恒温热源接触,而非系统保持恒温..

这促使我们思考是否有与卡诺循环中等温过程更加对应的过程来实现类卡诺循环过程。等温捷径\cite{Li2016}的提出为实现这个目标提供了契机,李耿等人给一个系统引入了辅助势,这个系统的演化本来是被一个含时哈密顿量所决定的,这个精心选择的辅助势可以使得系统在当前的哈密顿量下的状态,仿佛处于原哈密顿量的瞬时平衡态中一样,从而实现了“等温过程”,他称之为等温捷径.

于是,我们意欲构造一种由绝热捷径和等温捷径构成的类卡诺循环热机,并且根据随机热力学中的典型方法,研究该热机工作过程中的效率与功率。 与其他类型的热机,如涂展春构建的类卡诺热机\cite{Tu2013},Campo等人构建的奥托热机\cite{DelCampo2014}进行比较。
\section{经典热力学}

\qquad 经典热力学是从宏观的角度研究物质的。它所用的概念,如温度、密度、压强等,都是物质的宏观属性。这些都是不以物质的原子结构为基础就可以定义的概念。再根据宏观的观察和分析得到关于这些概念的经验性的规律和关系,经典热力学就是这样得以发展的。这些经验性的规律和关系被延伸为四个基本定律,即所谓的热力学第零定律、第一定律、第二定律、第三定律,以这些定律为基础就构成了经典热力学。热力学原理(及其基本定律)具有相当的普遍性,即它们不依赖于任何有关物质结构的特殊假设,因为它们根本不以物质的微观特性为基础。由于这种普遍性,热力学的应用范围非常广泛。以至于爱因斯坦说道\cite{schilpp1979albert}:“(热力学)是唯一普遍适用的物理理论,我确信,在这个理论的基本概念适用的范围内,它绝不会被推翻。”\footnote{It is the only physical theory of universal content, which I am convinced, that within the framework of applicability of its basic concepts will never be
overthrown.}

\subsection{经典热力学简史}

\qquad 经典热力学的历史最早可以追溯到克里克(Otto von Guericke),他在1650年建造和设计了世界上第一台真空泵,并借此演示了著名的马德堡半球实验(Magdeburg hemispheres)。Guericke的实验推翻亚里士多德(Aristotle)长期以来的假设,即 "大自然厌恶真空"。之后不久,英裔爱尔兰物理学家和化学家波义耳(Robert Boyle)得知了Guericke的设计,并在1656年与英国科学家胡克(Robert Hooke)合作,制造了一个气泵。\cite{partington1989short} 利用这个气泵,Boyle和Hooke注意到了压力、温度和体积之间的关系。随着时间的推移,\textbf{波义耳定律(Boyle's Law)}被发现,该定律指出:温度相同时,气体压力和体积成反比。

时间来到1824年,这一年“热力学之父”卡诺发表了\textit{Reflections on the Motive Power of Fire} (《关于火的动力的思考》),论述了热、功率、能量和热机效率。该书概述了卡诺热机、卡诺循环和动力之间的基本能量关系。它标志着热力学作为一门现代科学的开始。\cite{perrot1998z}

随后在19世纪50年代,热力学第一定律和第二定律被提出,主要出自威廉·兰金(William Rankine)、克劳修斯(Rudolf Clausius)和开尔文(William Thomson)的著作。接着,统计热力学的基础由麦克斯韦(James Clerk Maxwell)、玻尔兹曼(Ludwig Boltzmann)、普朗克、克劳修斯和吉布斯(J.Willard Gibbs)等物理学家建立。

1873-76年间,吉布斯发表了一系列的三篇论文,最著名的是\textit{ On the Equilibrium of Heterogeneous Substances}(《论异质物质的平衡》)\cite{connecticut1866transactions},他在论文中说明了如何利用图像分析包括化学反应在内的热力学过程,通过研究热力学系统的能量、熵、体积、温度和压力,可以确定一个过程是否会自发发生。\cite{sugi1993scientific} 另外,19世纪的杜赫姆(Pierre Duhem)也写了关于化学热力学的文章\cite{duhem1886potentiel}。 20世纪初,吉尔伯特·N·刘易斯(Gilbert N. Lewis)、梅尔·兰德尔(Merle Randall)\cite{lewis1923thermodynamics}和E·A·古根海姆(E. A. Guggenheim)\cite{guggenheim2002modern,guggenheim1967advanced}等化学家将吉布斯的方法应用到化学过程的分析中。从中也可以窥见热力学的适用性之广。

\subsection{经典热力学的基本概念}
\qquad 按爱因斯坦的说法,在热力学的基本概念适用范围内,热力学本身也是适用的。这一方面说明了热力学概念的运用之宽泛,另一方面也说明爱因斯坦并没有将热力学的适用范围无限扩大。不管如何,这都说明了热力学中的这些概念的重要性。而且现在,其中很大一部分概念已经渗透进其他学科、社会和生活的方方面面。
\begin{itemize}
    \item 系统\cite{Wu2010}:泛指个体的集合,这些个体之间通常有相互作用。在经典热力学中,一个系统通常指宏观系统,其包含的基本个体数的量级为$N_{\mathrm{A}}$。不与外界相互作用,包括不进行物质交换和热交换等的系统被称为\textbf{孤立系统}。
    \item 平衡态:在没有外界影响的条件下,如果一个系统的的状态不随时间变换,没有力学的、化学的变化和热传递,我们称这个系统达到了热力学平衡态。
    \item 态函数:对于一个处于热力学平衡态的系统,有若干的宏观性质,这些性质可以用宏观的量加以描述。比如\textbf{温度}、\textbf{体积}、\textbf{压强}、\textbf{自由能}、\textbf{焓}、\textbf{内能}、\textbf{熵},注意这些态函数并不都是独立的。
    \item 内能:内能是系统的态函数,可以看做系统所包含全部个体的平均动能和相互作用势能之和
    \item 熵:熵是系统的态函数,它和热力学第二定律息息相关。大概是热力学中最具影响力的概念,已经成为科学衡量无序和信息的基本标尺,在社会学中也有一定的应用。\cite{rifkin1980entropy}
\end{itemize}

\subsection{热力学基本定律}
\qquad 如前所述,热力学建立在四个基本定律之上。对于具体的系统,具体的问题,通常还要在加上一些唯象的假设,才能够解决相应的问题。
\begin{itemize}
    \item 热力学第零定律:如两个系统之间达到了热平衡态,则两个系统必有一相同的态函数。这个态函数其实就是温度。\cite{Wu2010}热力学第零定律也常常被叙述为:如果两个系统分别与第三个系统达到热平衡态,则这两个系统之间也达到了热平衡态。
    \item 热力学第一定律:热力学第一定律描述了能量在转换与转移过程中的守恒。设一个系统从外界吸收热量$Q$,外界对它做功$W$,内能增加$\Delta U$,则必有$\Delta U = Q + W$。这个定律是由经验而来的,因为所有妄图无中生有产生能量的企图都宣告失败。
    \item 热力学第二定律:第一定律对能量的转化和转移过程除了作了守恒限制外,并没有做其他限制。但就经验来看,有些过程虽然没有违反第一定律,但依然不可能发生。根据这些,我们可提炼出热力学第二定律,它说明系统演化的方向性。热力学第二定律有很多不同的等价的表述方式,如克劳修斯表述:热量不能自发的从低温物体传递到高温物
    体;开尔文表述:不可能从单一热源中吸收热量,并全部用来做功,而不产生其他影响。\cite{Wang2013}

    根据热力学第二定律可以定义系统的一个态函数——熵,两个平衡态的熵差定义为$\Delta S = S_{A} - S_{B} = \int_{A}^{B} \frac{\dbar Q{\mathrm{rev}}}{T}$,其中下标rev代表可逆过程。根据熵的定义,由热力学第二定律可以证明孤立系统的熵永不减少,即$\md S \ge 0$。这也是热力学第二定律的数学表述。

    到19世纪末50年代,在由Maxwell和Boltzmann等人建立的统计热力学中,给予了熵以微观的理解——系统的无序度。按此观点,热力学第二定律是一个概率性的定律。
    \item 热力学第三定律:一个凝聚系统,在可逆等温过程中的熵变,在热力学温度$T$趋近于$0$时也趋近于零。第三定律有如下著名推论:不可能通过有限步骤使一个物体的温度达到热力学零度。有了热力学第三定律我们就可以把系统的熵确切的定义出来\cite{Wang2013},而不是只定义熵变。
\end{itemize}

\section{随机热力学}
\qquad 经典热力学主要研究系统平衡态下的性质和平衡态间的转化,对于处于非平衡态下的系统的性质和不可逆过程中,经典热力学能回答我们的并不多。对于偏离平衡态不多的非平衡态的性质,在20世纪三四十年代,普里高津\cite{prigogine1965introduction}和德格鲁特\cite{de2013non}等人建立了线性不可逆热力学。它现在已经发展成为一个成熟的理论,也解决了非常多的非平衡态中的问题,比如温差电效应等。\cite{Wang2013}本文的重点并非线性不可逆热力学,在此不做介绍。

对于远离平衡态的情形,线性不可逆热力学依然无能为力。21世纪以来,生物科技和纳米技术的发展促进了人们对\textbf{小系统}的研究,与经典热力学研究的宏观系统不同,这种纳米尺度的小系统包含的粒子数要远小于$N_{\mathrm{A}}$。所以小系统的涨落效应和脱离平衡态的的程度都很大。1998年,Ken Sekimoto在研究用过阻尼郎之万方程描述的随机体系中,基于状态演化的随机相轨,给出了体系内能变化、输入功和吸收热等热力学变量的定义\cite{Sekimoto1998}。2005年,Udo Seifert进一步阐明,在受驱动布朗粒子中,体系的熵和熵产生同样可以基于单个随机相轨进行定义\cite{Seifert2005},至此随机热力学正式建立。它将经典热力学中的功、热和熵等概念扩展到了单个轨道的情形。下面我们以一维布朗粒子为例进行介绍。

\subsection{随机动力学}
\label{c1s3}
\qquad 根据牛顿第二定律,可以容易的写出处于一维势场中的布朗粒子的运动方程
\begin{equation}
    \frac{\md q}{\md t}=p, \quad \frac{\md p}{\md t}=-\frac{\partial U(q,\lambda)}{\partial q}-\gamma p+\xi(t)
    \label{eq1.1}
\end{equation}
这就是\textbf{郎之万方程(Langevin Equation)},其中$\gamma$是阻尼系数,$\xi(t)$是高斯白噪声。依赖于时间的势场$U(q,\lambda)$施加在布朗粒子上,$\lambda\equiv\lambda(t)$是外界控制的势场的参数。对于$\xi(t)$,我们有
\begin{equation}
    \left\langle\xi\left(t\right)\right\rangle_{\xi}=0,\quad \left\langle\xi\left(t_{1}\right) \xi\left(t_{2}\right)\right\rangle_{\xi}=2 \gamma T  \delta\left(t_{2}-t_{1}\right)
    \label{eq1.2}
\end{equation}
其中$\left\langle \cdots \right\rangle_{\xi}$是对随机变量$\xi(t)$的不同实现的平均,第二个式子等式右边的系数$2 \gamma T$是能量均分定理的要求。\cite{Reichl2016}

在本文中,我们更常用的是描述布朗粒子的另一个等价的方程——\textbf{福克-普朗克方程(Fokker–Planck Equation)}\cite{Reichl2016},福克-普朗克方程从另一个方面描述了布朗粒子,决定了布朗粒子的概率密度的演化。令$\rho_{\xi}(q,p,t)$为布朗粒子对于某一个随机力的实现$\xi(t)$的概率密度,而我们关心的实际上是$\rho(q,p,t)\equiv\langle \rho(q,p,t) \rangle_{\xi}$,即对随机力的实现$\xi(t)$的平均,这个量才是我们在实验中能够观测到的。它表明在实际情形下,我们在时刻$t$,位置$q+\md q$处找到动量为$p+\md p$的布朗粒子的概率为$\rho(q,p,t) \md q \md p$。福克-普朗克方程正是描述了这个量的演化。下面我们来从朗之万方程推导福克-普朗克方程

对于$\rho_{\xi}(q,p,t)$,有刘维尔定理(Liouville Theorem)\cite{Wang2013}
\begin{equation}
    \PP{\rho_{\xi}}{t} + \nabla \cdot \left(\rho_{\xi} \bm{v} \right)=0
    \label{eq1.3}
\end{equation}
其中$\bm{v}\equiv\left(\dot{q}\hat{\mathbf{q}} + \dot{p}\hat{\mathbf{p}}\right)$,算符$\nabla \equiv \hat{\mathbf{q}} \partial / \partial q+\hat{\mathbf{p}} \partial / \partial p$,$\hat{\mathbf{q}}$和$\hat{\mathbf{p}}$是单位方向向量。
将朗之万方程\eqref{eq1.1}代入上式\eqref{eq1.3}得到
\begin{equation}
    \frac{\partial \rho_{\xi}}{\partial t}=-\hat{L}_{0} \rho_{\xi}-\hat{L}_{1} \rho_{\xi}
    \label{eq1.4}
\end{equation}
其中微分算符$\hat{L}_{0}$和$\hat{L}_{1}$的定义为
\begin{equation}
    \hat{L}_{0}=p \frac{\partial}{\partial q}-{\gamma}-{\gamma} p \frac{\partial}{\partial p}-\frac{\partial U}{\partial q} \frac{\partial}{\partial p},\quad \hat{L}_{1}= \xi(t) \frac{\partial}{\partial p}
    \label{eq1.5}
\end{equation}

引入概率密度$\sigma(t)$,使得
\begin{equation}
    \rho_{\xi}=\mathrm{e}^{-\hat{L}_{0} t} \sigma
    \label{eq1.6}
\end{equation}
根据式\eqref{eq1.4}有
\begin{equation}
    \frac{\partial \sigma(t)}{\partial t}=-\hat{V}(t) \sigma(t)
    \label{eq1.7}
\end{equation}
其中$\hat{V}(t)=\mathrm{e}^{+\hat{L}_{0} t} \hat{L}_{1}(t) \mathrm{e}^{-\hat{L}_{0} t}$。方程\eqref{eq1.7}有形式解
\begin{equation}
    \sigma(t)=\exp \left(\int_{0}^{t} \mathrm{~d} t^{\prime} \hat{V}\left(t^{\prime}\right)\right) \sigma(0)
    \label{eq1.8}
\end{equation}
利用幂级数展开可将上式\eqref{eq1.8}写成
\begin{equation}
    \sigma(t)=\left[\sum_{n=0}^{\infty} \frac{(-1)^{n}}{n !}\left(\int_{0}^{t} \mathrm{~d} t^{\prime} \hat{V}\left(t^{\prime}\right)\right)^{n}\right] \sigma(0) .
    \label{eq1.9}
\end{equation}
现在我们取等式\eqref{eq1.9}两边的平均$\langle \cdots \rangle_{\xi}$。根据对高斯白噪声的Wick定理\cite{Reichl2016},幂级数将只留下偶次幂项
\begin{equation}
    \begin{split}
        \langle\sigma(t)\rangle_{\xi}&=\left[\sum_{n=0}^{\infty} \frac{1}{2 n !}\left\langle\left(\int_{0}^{t} \mathrm{~d} t^{\prime} \hat{V}\left(t^{\prime}\right)\right)^{2 n}\right\rangle_{\xi}\right] \sigma(0)\\
        &=\left[\sum_{n=0}^{\infty} \frac{1}{n !}\left(\frac{1}{2} \int_{0}^{t} \mathrm{~d} t_{2} \int_{0}^{t} \mathrm{~d} t_{1}\left\langle\hat{V}\left(t_{2}\right) \hat{V}\left(t_{1}\right)\right\rangle_{\xi}\right)^{n}\right] \sigma(0)\\
        &=\exp \left(\frac{1}{2} \int_{0}^{t} \mathrm{~d} t_{2} \int_{0}^{t} \mathrm{~d} t_{1}\left\langle\hat{V}\left(t_{2}\right) \hat{V}\left(t_{1}\right)\right\rangle_{\xi}\right) \sigma(0)
    \end{split}
    \label{eq1.10}
\end{equation}
现在来计算上式\eqref{eq1.10}中的积分
\begin{equation}
    \begin{split}
    &\frac{1}{2} \int_{0}^{t} \mathrm{d} t_{2} \int_{0}^{t} \mathrm{d} t_{1}\left\langle\hat{V}\left(t_{2}\right) \hat{V}\left(t_{1}\right)\right\rangle_{\xi} \\
    =&\gamma T \int_{0}^{t} \mathrm{d} t_{2} \int_{0}^{t} \mathrm{~d} t_{1} \delta\left(t_{2}-t_{1}\right) \mathrm{e}^{+\hat{L}_{0} t_{2}} \frac{\partial}{\partial p} \mathrm{e}^{-\hat{L}_{0}\left(t_{2}-t_{1}\right)} \frac{\partial}{\partial p} \mathrm{e}^{-\hat{L}_{0} t_{1}} \\
    =&\gamma T \int_{0}^{t} \mathrm{d} t_{1} \mathrm{e}^{+\hat{L}_{0} t_{1}} \frac{\partial^{2}}{\partial p^{2}} \mathrm{e}^{-\hat{L}_{0} t_{1}}   
    \end{split}
    \label{eq1.11}
\end{equation}
将上式\eqref{eq1.11}代入式\eqref{eq1.10},并对式\eqref{eq1.10}取对时间$t$的偏微分可得
\begin{equation}
    \frac{\partial\langle\sigma(t)\rangle_{\xi}}{\partial t}=\gamma T \mathrm{e}^{+\hat{L}_{0} t} \frac{\partial^{2}}{\partial p^{2}} \mathrm{e}^{-\hat{L}_{0} t}\langle\sigma(t)\rangle_{\xi}
    \label{eq1.12}
\end{equation}
有了这个结果,我们就能得到$\rho(q,p,t)$的运动方程,注意到$\rho=\langle\rho_{\xi}\rangle_{\xi}=\mathrm{e}^{-\hat{L}_{0} t} \langle\sigma\rangle_{\xi}$,于是不难得到
\begin{equation}
    \frac{\partial \rho}{\partial t}=-\hat{L}_{0}\rho+\mathrm{e}^{-\hat{L}_{0} t} \frac{\partial\langle\sigma(t)\rangle_{\xi}}{\partial t}=-\hat{L}_{0}\rho+\gamma T \frac{\partial^{2} \rho}{\partial p^{2}}
    \label{eq1.13}
\end{equation}
结合上式\eqref{eq1.13}和式\eqref{eq1.5}可以得到
\begin{equation}
    \frac{\partial \rho}{\partial t}=-v \frac{\partial \rho}{\partial q}+\frac{\partial}{\partial p}\left[\left({\gamma} v+ \PP{U}{q}\right) \rho\right]+\gamma T \frac{\partial^{2} \rho}{\partial p^{2}}
    \label{eq1.14}
\end{equation}
这就是福克-普朗克方程。很重要的是我们注意到福克-普朗克方程满足概率守恒,这一点将它写为连续性方程的形式就清楚了。
\begin{equation}
    \PP{\rho}{t} + \nabla \cdot \bm{J} = 0
\label{eq12.57}
\end{equation}
其中
\begin{equation}
\bm{J} \equiv \rho  p  \hat{\mathbf{q}}-\rho\left(\gamma p+\frac{\partial U}{\partial q} + \frac{\gamma}{\rho \beta} \frac{\partial \rho}{\partial p}\right) \hat{\mathbf{p}}
\label{eq12.57.5}
\end{equation}
算符$\nabla \equiv \hat{\mathbf{q}} \partial / \partial q+\hat{\mathbf{p}} \partial / \partial p$,$\hat{\mathbf{q}}$和$\hat{\mathbf{p}}$是单位方向向量。

对于阻尼很大的情况,即过阻尼情况,我们可以忽略惯性的影响得到过阻尼情况下的郎之万方程
\begin{equation}
    0=-\frac{\partial U}{\partial q}-\gamma \frac{\md q}{\md t}+\xi(t)
    \label{eq1.17}
\end{equation}
对应的也有过阻尼的福克-普朗克方程
\begin{equation}
    \frac{\partial \rho}{\partial t} + \frac{\partial J}{\partial q}=0,
    \label{eq2.45}
\end{equation}
其中
\begin{equation}
    J=-\frac{1}{\gamma} \left(\frac{\partial U}{\partial q} \rho+\frac{1}{\beta} \frac{\partial \rho}{\partial q}\right)
    \label{eq2.45.5}
\end{equation}
上面的郎之万方程和相应的福克-普朗克方程都有自己适用的时间尺度,欠阻尼郎之万方程\eqref{eq3.1}适用的时间尺度要远大于环境中分子碰撞的时间尺度而远小于惯性的弛豫时间尺度$\tau_{\mathrm{c}}\equiv1/\gamma$;过阻尼郎之万方程适用的时间尺度则要远大于$\tau_{\mathrm{c}}$。\cite{Sekimoto2010}这一点在第\ref{c3e2}节中计算欠阻尼类卡诺热机的效率时将起到重要作用。

\subsection{随机能量学}
\label{c1s1.3.2}
\qquad 现在我们考虑为布朗粒子定义和经典热力学中对应的\textbf{吸收热}、\textbf{输入功}和\textbf{能量差}。环境对布朗粒子施加的力为$-\gamma p + \xi(t)$,那么粒子对环境有反作用力$-(-\gamma p + \xi(t))$,对于布朗粒子的一个轨道,当粒子的位移为$\md q$时,其对环境做的功为$-(-\gamma p+\xi(t)) \circ \md q$\footnote{"$\circ$"表示Stratonovich类型的乘积,它允许我们使用常规的微分运算。\cite{Gardiner,Sekimoto2010}}从环境的角度看,环境向粒子传递了$(-\gamma p+\xi(t)) \circ \md q$的能量。Sekimoto将它\cite{Sekimoto1997,Sekimoto2010}定义为吸收热
\begin{equation}
    \dbar \bar{q} \equiv (-\gamma p+\xi(t)) \circ \md q
    \label{eq1.20}
\end{equation}
考虑到能量$e=\frac{1}{2}p^2 + U(q,\lambda)$,再结合郎之万方程\eqref{eq1.1},不难得到
\begin{equation}
    \begin{split}
        \dbar \bar{q} & \equiv\left(\frac{\md p}{\md t}+\frac{\partial U}{\partial q}\right) \circ \md q \\
        &=\frac{\md p}{\md t} \circ p \md t+\md U(q, \lambda)-\frac{\partial U}{\partial \lambda} \md \lambda \\
        &=\md\left(\frac{p^{2}}{2}\right)+d U(x, \lambda)-\frac{\partial U}{\partial \lambda} \md \lambda
    \end{split}
    \label{eq1.21}
\end{equation}
易得
\begin{equation}
    \md e=\md\left(\frac{p^{2}}{2}+U(q, \lambda)\right)=\dbar q+\frac{\partial U}{\partial \lambda} \md \lambda
    \label{eq1.22}
\end{equation}
结合经典热力学中的热力学第一定律,根据上式\eqref{eq1.22}我们可以定义输入功
\begin{equation}
    \dbar w \equiv \frac{\partial U}{\partial \lambda} \md \lambda
    \label{eq1.23}
\end{equation}

自此我们定义小系统中的热、功和能量差,这为我们讨论热机的功率和效率做好了准备。