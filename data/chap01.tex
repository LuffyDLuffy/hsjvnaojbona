\chapter{绪论:从经典热力学到随机热力学}
热机的发明和使用对人类的生产生活产生了重大的影响,第一次工业革命和第二次工业革命都和热机的发展有紧密的关系. 一直以来,特别是近些年来,由于能源短缺的问题愈发突出,研究热机的效率问题也一直吸引着一大批研究者。

\section{引言:热机的最大功率与效率问题}
\quad 早在1824,卡诺就指出(基于热质说)\cite{2005}:工作在相同高温热源和相同低温热源的所有热机中,以可逆热机的效率最高,而且这些可逆热机的效率都相同. 为$\eta _{\text{C}}=1-\frac{T_1}{T_2}$ 。这个问题似乎就这么解决了,然而必须注意的是可逆热机的要求相当于要求工作物质是准静态的. 也就是说,要想严格意义上达到卡诺热机的效率,热机一个循环的工作时间应当是无限长,这样的热机功率是0.

我们当然不可能接受功率为0的热机,所以得到热机保持在某一功率下(特别是最大功率)的最大效率成了摆在我们面前的问题,而这个问题促进了有限时间热力学的诞生. 

1975年,Curzon和Ahlborn研究了内可逆热机\cite{Curzon1975},该热机可在有限时间内完成一个循环. 利用Newton热运输规律,得到了该热机在最大功率下的效率为$\eta _{\mathrm{CA}}=1-\sqrt{T_{\mathrm{c}}/T_{\mathrm{h}}}=1-\sqrt{1-\eta _{\mathrm{C}}}$ ,其中$\eta _{\mathrm{C}}$为卡诺效率。

2008年,T. Schmiedl 和 U. Seifert提出了布朗随机热机模型\cite{Schmiedl2008},该热机模型利用谐振子势场驱动布朗粒子做功,并得到了其在最大功率下的效率:$\eta_{2}=\frac{2 \eta_{0}}{4-\eta_{0}}$

同年,涂展春推导出费曼棘轮热机\cite{Tu2008}在最大功率下的效率$\eta _{\text{F}}=\frac{\eta _{\text{C}}^{2}}{\eta _{\text{C}}-\left( 1-\eta _{\text{C}} \right) \ln \left( 1-\eta _{\text{C}} \right)}$. \cite{Tu2020}

……



同时涂展春\cite{Tu2008}在小温差条件下,即$\eta _{\text{C}}\rightarrow 0$. 将上述热机的效率按照 展开. 发现这些效率到二阶项都相同的,它们都普遍地满足$\eta _{\text{U}}=\frac{\eta _{\text{C}}}{2}+\frac{\eta _{\text{C}}^{2}}{8}+O\left( \eta _{\text{C}}^{3} \right)$. 这个规律在其他热机中也得到了验证,但也有一些热机不满足这个关系. 这引发了研究者对相关问题的思考,涂展春在紧耦合热机的范畴内对其中的原因进行了解释.\cite{Tu2020}

在这些形形色色的有限时间热机中,由Seifert等人提出的随机热机\cite{Schmiedl2008}占据一席之地. 随机热机通常是利用外势驱动与热源接触的布朗粒子做功. 其中的具体的实现过程可以有很大的不同,对应着不同的随机热机. 在这些不同的随机热机中, Campo等人利用绝热捷径构建的奥托热机\cite{DelCampo2014}引人注目. 而后,Deng等人发现绝热捷径能够提高奥托热机的效率,不论是经典的还是量子的.\cite{Deng2013} 涂展春在Schmiedl和Seifert工作\cite{Schmiedl2008}的基础上,考虑了他们在过阻尼情况下忽略的惯性的影响,构建了一种类卡洛热机\cite{Tu2013}. 惊讶地发现这种随机热机的效率等于Curzon和Ahlborn构建的内可逆热机\cite{Curzon1975}的效率 . 在涂展春构建的热机中\cite{Tu2013},布朗粒子在与时间相关的谐振子$U=\lambda ^2\left( t \right) x^2/2$的控制下,经历了类卡诺循环. 其中与卡诺循环中绝热过程对应的就是前面提到的绝热捷径,但与卡诺循环中等温过程对应的“等温过程”,却只是粒子与恒温热源接触,而非系统保持等温. 

这促使笔者思考是否有与卡诺循环中等温过程更加对应的过程,以实现类卡诺循环. 等温捷径\cite{Li2016}的提出为实现这个目标提供了契机,李耿等人给一个系统引入了辅助势,这个系统的演化本来是被一个含时哈密顿量所决定的,这个精心选择的辅助势可以使得系统在当前的哈密顿量下的状态,仿佛处于原哈密顿量的瞬时平衡态中一样,从而实现了“等温过程”,我们称之为等温捷径.

综上,笔者欲构造由绝热捷径和等温捷径构成的类卡诺循环,根据随机热力学和非平衡热力学中的典型方法,研究该热机工作过程中的功、熵、能量损失等参数,并考察它的效率与功率. 并与其他类型的热机,特别是涂展春构建的类卡洛热机\cite{Tu2013},Campo等人构建的奥托热机\cite{DelCampo2014}进行比较. 考虑到Deng等人发现绝热捷径能够提高奥托热机的效率\cite{Deng2013},我们也期望等温捷径的引入能进一步提高热机的效率


\section{经典热力学}

\qquad 经典热力学是从宏观的角度研究物质的。它所用的概念,如温度、密度、压强等,都是物质的宏观属性。这些都是不以物质的原子结构为基础就可以定义的概念。再根据宏观的观察和分析得到关于这些概念的经验性的规律和关系,经典热力学就是这样得以发展的。这些经验性的规律和关系被延伸为四个个基本定律,即所谓的热力学第零定律,第一定律,第二定律,第三定律,以这些定律为基础就构成了所谓的经典热力学。热力学原理(即其基本定律)具有相当的普遍性,即它们不依赖于任何有关物质结构的特殊假设,因为它们根本不以物质的原子特性为基础。由于这种普遍性,热力学的应用范围非常广泛。以至于爱因斯坦说道\cite{schilpp1979albert}:“(热力学)这是唯一普遍适用的物理理论,我确信,在这个理论的基本概念适用的范围内,它绝不会被推翻。”(It is the only physical theory of universal
content, which I am convinced, that within the framework
of applicability of its basic concepts will never be
overthrown.)

经典热力学的历史最早可以追溯到Otto von Guericke(克里克),他在1650年建造和设计了世界上第一台真空泵,并用此演示了著名的Magdeburg hemispheres(马格德堡半球实验)。Guericke的实验推翻Aristotle(亚里士多德)长期以来的假设,即 "大自然厌恶真空"。之后不久,英裔爱尔兰物理学家和化学家Robert Boyle(波义耳)得知了Guericke的设计,并在1656年与英国科学家Robert Hooke(胡克)合作,制造了一个气泵。\cite{partington1989short} 利用这个气泵,Boyle和Hooke注意到压力、温度和体积之间的关系。随着时间的推移,\textbf{Boyle's Law(波义耳定律)}被发现,该定律指出:温度相同时,气体压力和体积成反比。

时间来到1824年,这一年"热力学之父 "Sadi Carnot(卡诺)发表了\textit{Reflections on the Motive Power of Fire} (《关于火的动力的思考》),这是一本关于热、功率、能量和发动机效率的论述。该书概述了卡诺热机、卡诺循环和动力之间的基本能量关系。它标志着热力学作为一门现代科学的开始。\cite{perrot1998z}

随后在19世纪50年代,热力学第一定律和第二定律同时被提出,主要出自William Rankine(威廉·兰金)、Rudolf Clausius(克劳修斯)和William Thomson(开尔文勋爵)的著作。接着,统计热力学的基础由James Clerk Maxwell(麦克斯韦)、Ludwig Boltzmann(玻尔兹曼)、Max Planck(普朗克)、Rudolf Clausius(克劳修斯)和J.Willard Gibbs(吉布斯)等物理学家建立。

1873-76年间,美国物理学家吉布斯发表了一系列的三篇论文,最著名的是\textit{ On the Equilibrium of Heterogeneous Substances}(《论异质物质的平衡》)\cite{connecticut1866transactions},他在论文中说明了如何利用图像分析包括化学反应在内的热力学过程,通过研究热力学系统的能量、熵、体积、温度和压力,可以确定一个过程是否会自发发生。\cite{sugi1993scientific} 另外,19世纪的Pierre Duhem(皮埃尔-杜赫姆)也写了关于化学热力学的文章\cite{duhem1886potentiel}。 20世纪初,Gilbert N. Lewis(吉尔伯特·N·刘易斯)、Merle Randall(梅尔·兰德尔)\cite{lewis1923thermodynamics}和E. A. Guggenheim(E·A·古根海姆)\cite{guggenheim2002modern,guggenheim1967advanced}等化学家将吉布斯的方法应用到化学过程的分析中。从中也可以窥见热力学的适用性之广。


热力学中的一个重要概念是热力学系统,它是所研究的宇宙中一个精确定义的区域。宇宙中除了系统以外的一切都被称为周围环境。一个系统被一个边界与宇宙的其余部分隔开,这个边界可能是物理的或名义的,但作用是将系统限制在一个有限的体积内。边界的部分通常被描述为墙,它们有各自定义的 "渗透性"。根据各自的渗透性,能量作为功、热或物质在系统和周围环境之间的传递,通过墙壁进行。

跨越边界的物质或能量需要在能量平衡方程中进行计算,以影响系统内部能量的变化。墙体所包含的体积可以是单个原子共振能量周围的区域,如马克斯-普朗克在1900年定义的;可以是蒸汽机中的蒸汽或空气体,如萨迪-卡诺在1824年定义的。系统也可以只是一个核素(即夸克的系统),如量子热力学中的假设。当采用较宽松的观点,放弃热力学平衡的要求时,系统可以是热带气旋的主体,如1986年Kerry Emanuel在大气热力学领域的理论,也可以是黑洞的事件视界。

边界有四种类型:固定的、可动的、实的和虚的。例如,在发动机中,固定边界意味着活塞被锁定在其位置上,在此范围内可能会发生一个恒定的体积过程。如果允许活塞移动,该边界是可动的,而气缸和气缸盖的边界是固定的。对于封闭系统来说,边界是实的,而对于开放系统来说,边界往往是虚的。在喷气式发动机的情况下,在发动机的进气口处可能会假设一个固定的虚边界,沿壳体表面有固定的边界,在排气口处有第二个固定的虚边界。

一般来说,热力学区分了三类系统,以允许什么东西越过它们的边界来定义。

热力学系统的相互作用
系统类型 质量流量 工作量 热量
开启 绿色刻度线 绿色刻度线 绿色刻度线
关闭 红色 X 绿色 勾 绿色 勾
热隔离 红色 X 绿色 tick 红色 X
机械隔离 红色 X 红色 X 绿色 剔号
隔离的红X红X红X
随着时间的推移,在一个孤立的系统中,压力、密度和温度的内部差异趋于平衡。在一个系统中,所有的平衡过程都已完成,可以说是处于热力学平衡状态。

一旦处于热力学平衡状态,根据定义,一个系统的特性在时间上是不变的。处于平衡状态的系统比不处于平衡状态的系统简单得多,也容易理解。在分析一个动态热力学过程时,往往简化假设过程中的每一个中间状态都处于平衡状态,产生的热力学过程发展缓慢,以至于每一个中间步骤都是平衡状态,被称为可逆过程。

当一个系统在给定条件下处于平衡状态时,称其处于确定的热力学状态。系统的状态可以用一些不依赖于系统到达其状态的过程的状态量来描述。根据系统大小变化时的变化情况,它们被称为密集变量或广泛变量。系统的特性可以用一个状态方程来描述,它规定了这些变量之间的关系。状态可以认为是在设定的变量数量保持不变的情况下对系统的瞬时定量描述。

热力学过程可以定义为一个热力学系统从初始状态到最终状态的能量演化过程。它可以用过程量来描述。通常,每个热力学过程在能量特性上与其他过程的区别在于哪些参数,如温度、压力或体积等是固定不变的;此外,将这些过程分成若干对,其中每个保持不变的变量都是共轭对中的一个成员。

几个常用的研究的热力学过程是

绝热过程:发生时没有热能的损失或增加。
等焓过程:在恒定的焓值下发生。
等熵过程:一个可逆的绝热过程,在一个恒定的熵下发生。
等压过程:在恒定压力下发生。
等容过程:在恒定体积下发生(也称为等容/等体积)。
等温过程:在恒温下进行。
稳态过程:发生时内能不发生变化。
\section{随机热力学}

