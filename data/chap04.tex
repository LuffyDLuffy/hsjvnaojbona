\chapter{总结与展望}
在本篇文章中,我们利用绝热捷径和等温捷径构建了一个类卡诺热机模型,再结合随机热力学的知识,计算了该热机的效率与功率。通过借鉴T. Schmiedl 和 U. Seifert\cite{Schmiedl2008}等人处理微观随机热机的方法,利用求多元函数极值和求泛函极值的办法,在过阻尼和欠阻尼的情形下,得到了构建的热机在最大功率时的效率。

而在绝热捷径中构造生成元的方法,除了可以应用在量子和经典动力学之外,也可应用于统计力学中,用于估算自由能的变化。在统计力学中,为了减少或者消除对有限时间过程中数字模拟的不可逆性,反绝热的度规缩放\cite{Miller2000}和场流\cite{Vaikuntanathan2008}也被构造出来。

对于等温捷径,由Le Cunuder和他的合作者所做的实验\cite{LeCunuder2016}说明了辅助势\eqref{eq2.61}是能够在实验上实现的,而辅助势\eqref{eq2.62}里的交叉项$qp$很类似于绝热捷径中的反绝热哈密顿量\cite{Jarzynski2013,Deffner2014,DelCampo2013}。对于一大类由绝热捷径控制的量子多体系统,del Campo提出了一种可能的实验方案\cite{DelCampo2013}来实现交叉项。这种方案不需要知道微观系统的详细信息,通过适当的正则变换,他得到了另一个表象,而在这个表象下没有交叉项。这个方案为我们实现辅助势\eqref{eq2.62}提供了思路。

在过阻尼情形下,我们发现构建的热机在最大功率时的效率为${\eta_{\mathrm{c}}}/\left[3-\left(\eta_{\mathrm{c}}+\sqrt{1-\eta_{\mathrm{c}}}\right)\right]$,这个效率符合效率的二阶普适性。但不同于T. Schmiedl 和 U. Seifert提出了微观随机热机模型\cite{Schmiedl2008},相比之下,我们构造的热机没有$\eta_{\mathrm{c}}$的三阶项。

而在欠阻尼情形下,热机在最大功率时的效率和涂展春构建的类卡诺循环热机\cite{Tu2013}一样,为$\eta_{\M{CA}}$。


我们的热机模型相对于涂展春构建的热机模型\cite{Tu2013}有一点不同。对于热机循环中的两个 “等温”过程,我们是利用等捷径来实现的,而涂展春是将粒子与恒温热浴接触。不过我们的结果也支持了涂展春的结论\cite{Tu2013}——恒定的有效温度的假设并不是实现Curzon-Ahlborn效率的必要条件。


再看看随机热机在实验中的可行性\cite{Tu2013},2012年,Blickle和Bechinger\cite{Blickle2012}通过使用时间相关的激光阱控制直径为2.94~$\M{\mu m}$的单个胶体粒子,证明了微观上斯特林热机的实验实现。对于直径为$\M{\mu m}$的颗粒,当我们以s为时间尺度进行观察时,其惯性效应太小,无法检测到,这对应于我们的过阻尼情形。为了增强惯性效应的相对强度,应该将时间分辨率提高到$\M{\mu s}$级。实验上\cite{Li2013,Huang2011}已经做到了这一点,他们对$1 ~ \M{\mu m}$的二氧化硅粒子的时间分辨率达到了$10 ~ \M{n m}$。