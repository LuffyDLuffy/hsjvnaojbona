\chapter{绝热捷径与等温捷径构成的类卡诺循环热机}
\section{布朗粒子的能量学}
\qquad 在正式开始讨论我们构建的热机之前,为了方便讨论热机的效率、熵变等问题,让我们先花一点时间讨论布朗粒子的能量学。
\subsection{过阻尼布朗粒子的能量学}
\R{李耿博士毕业论文,第一章随机能量学可作为参考}

考虑一个和温度为$T$的热浴接触的一维布朗粒子。它被施加一个依赖于时间的外势$U(q,p,t) = U_0 (q,\lambda(t)) + U_1 (q,t)$,其中考虑到了在绝热捷径和等温捷径中我们施加的辅助势$U_1$。在过阻尼情况下,粒子的惯性的影响可以忽略,哈密顿量为
\begin{equation}
    H=U_0 (q,\lambda(t)) + U_1 (q,p,t)
    \label{eq31.1}
\end{equation}
哈密顿量的全微分为
\begin{equation}
    \M{d} H=\left(\dot{q} \frac{\partial U_0}{\partial q} + \dot{q} \PP{U_1}{q}  \right) \M{d} t+\left(\dot{\lambda} \frac{\partial U_0}{\partial \lambda} + \PP{U_1}{t} \right) \M{d} t
    \label{eq31.2}
\end{equation}
这提示我们定义沿着相轨迹$(q(t), p(t))$的能量差\cite{Tu2013}
\begin{equation}
     \Delta e \equiv H\left(t_{f}\right)-H\left(t_{i}\right),
     \label{eq31.3}
\end{equation}
输入功
\begin{equation}
    w \equiv \int_{t_{i}}^{t_{f}} \M{d} t \left( \dot{\lambda} \frac{\partial U_0}{\partial \lambda} + \PP{U_1}{t} \right)
    \label{eq31.4}
\end{equation}
这也与传统的随机热力学对轨道功的定义相符合。\cite{Sekimoto2010,Jarzynski1997,Sekimoto_1997}。还有吸收的热量
\begin{equation}
    q \equiv \int_{t_{i}}^{t_{f}} \M{d} t \dot{q} \frac{\partial U}{\partial q} .
    \label{eq31.5}
\end{equation}
相轨连接了初时刻$t_i$对应的相点$(q_i , p_i )$和末时刻$t_f$对应的相点$(q_f , p_f )$。显然,对于每条相轨有
\begin{equation}
    \Delta e = w + q
    \label{eq31.6}
\end{equation}

而系统的分布函数$\rho (q,p,t)$由的福克-普朗克-克拉马斯方程\eqref{eq2.45}决定,在知道了$\rho (q,p,t)$之后就可以计算以上式子\eqref{eq31.4}-\eqref{eq31.6}的系综平均,这和文献\cite{Seifert2005,Shizume1995,Bizarro2011}的步骤是类似的。得到能量差和输入功的系综平均为
\begin{equation}
    \Delta E \equiv\langle\Delta e\rangle=\left.\int \M{d} q \int (H \rho)\right|_{t_{i}} ^{t_{f}} ,
    \label{eq31.7}
\end{equation}
和
\begin{equation}
    W \equiv\langle w\rangle=\int_{t_{i}}^{t_{f}} \M{d} t \int \M{d} q \left[\rho \left( \dot{\lambda} \frac{\partial U_0}{\partial \lambda} + \PP{U_1}{t} \right) \right].
    \label{eq31.8}
\end{equation}

对于吸收的热的系综平均,由于存在$\dot{q}$,我们不好直接积分,需要做进一步的推导。系综平均为
\begin{equation}
    Q=\int_{t_i}^{t_f}\left\langle\dot{q} \frac{\partial U}{\partial q} \right\rangle \mathrm{d} t
    \label{eq31.9}
\end{equation}
现在分两步对上式\eqref{eq31.9}进行平均。

第一步,对系综中在$t$时刻经过$q$的轨道进行平均\R{(不懂至以下)},得到
\begin{equation}
    \langle\dot{x} \mid x, p, t\rangle=\frac{J}{\rho}
    \label{eq31.10}
\end{equation}
其中$J$是式\eqref{eq2.45.5}所定义的流。

第二步,利用分布函数$\rho(q,p,t)$对所有的$q$和$p$进行系综平均。于是,系统与热浴的热交换为
\begin{equation}
    \begin{split}
        Q&=\int_{t_i}^{t_f} \mathrm{d} t \int \mathrm{d} q \left(J \frac{\partial U}{\partial q}\right)\\
         &=\int_{t_i}^{t_f} \mathrm{d} t \int \mathrm{d} q \left[-\frac{1}{\gamma}  \frac{\partial U}{\partial q} \left(\frac{\partial U}{\partial q} \rho+\frac{1}{\beta} \frac{\partial \rho}{\partial q}\right) \right ]
    \end{split}   
    \label{eq31.11}
\end{equation}

% 现在我们将式\eqref{eq31.7},\eqref{eq31.8},\eqref{eq31.12}提到的计算能量差$\Delta E$、输入功$W$、吸收热$Q$的方法,应用于此模型中。对于过阻尼布朗粒子的等温捷径,注意到根据式\eqref{eq2.55},$U=U_0 + U_1 = q^2/{2 \lambda(t)^2}  -{\gamma \dot{\lambda}(t)  q^{2}}/{2 \lambda(t)}$,而根据\eqref{eq2.46}的假设,$\rho(q, t)= \mathrm{e}^{\beta\left[F(\lambda(t))-q^2/{2 \lambda(t)^2}\right]},\ F(\lambda) \equiv-\beta^{-1} \ln \left[\int_{-\infty}^{+\infty} e^{-\beta q^2/{2 \lambda(t)^2}} \M{d} q \right]$。不难得到,能量差

\subsection{欠阻尼布朗粒子的能量学}
\qquad 考虑一个和温度为$T$的热浴接触的一维布朗粒子。它被施加一个依赖于时间的外势$U(q,p,t) = U_0 (q,\lambda(t)) + U_1 (q,p,t)$,其中考虑到了在绝热捷径和等温捷径中我们施加的辅助势包含动量$p$。在欠阻尼情况下,粒子的惯性的影响不能忽略,哈密顿量为
\begin{equation}
    H=\frac{p^{2}}{2}+U_0 (q,\lambda(t)) + U_1 (q,p,t)
    \label{eq3.1}
\end{equation}
哈密顿量的全微分为
\begin{equation}
    \M{d} H=\left(\dot{p} p+\dot{q} \frac{\partial U_0}{\partial q} +  \dot{p} \PP{U_1}{p} + \dot{q} \PP{U_1}{q}  \right) \M{d} t+\left(\dot{\lambda} \frac{\partial U_0}{\partial \lambda} + \PP{U_1}{t} \right) \M{d} t
    \label{eq3.2}
\end{equation}
这提示我们定义沿着相轨迹$(q(t), p(t))$的能量差\cite{Tu2013}
\begin{equation}
     \Delta e \equiv H\left(t_{f}\right)-H\left(t_{i}\right),
     \label{eq3.3}
\end{equation}
输入功
\begin{equation}
    w \equiv \int_{t_{i}}^{t_{f}} \M{d} t \left( \dot{\lambda} \frac{\partial U_0}{\partial \lambda} + \PP{U_1}{t} \right)
    \label{eq3.4}
\end{equation}
这也与传统的随机热力学对轨道功的定义相符合。\cite{Sekimoto2010,Jarzynski1997,Sekimoto_1997}。还有吸收的热量
\begin{equation}
    q \equiv \int_{t_{i}}^{t_{f}} \M{d} t \left(\dot{p} p+\dot{q} \frac{\partial U}{\partial q} +  \dot{p} \PP{U}{p}\right).
    \label{eq3.5}
\end{equation}
相轨连接了初时刻$t_i$对应的相点$(q_i , p_i )$和末时刻$t_f$对应的相点$(q_f , p_f )$。显然,对于每条相轨有
\begin{equation}
    \Delta e = w + q
    \label{eq3.6}
\end{equation}

而系统的分布函数$\rho (q,p,t)$由推广的福克-普朗克-克拉马斯方程\eqref{eq2.57}决定。在知道了$\rho (q,p,t)$之后就可以计算以上式子\eqref{eq3.4}-\eqref{eq3.6}的系综平均,这和文献\cite{Seifert2005,Shizume1995,Bizarro2011}的步骤是类似的。得到能量差和输入功的系综平均为
\begin{equation}
    \Delta E \equiv\langle\Delta e\rangle=\left.\int \M{d} q \int \M{d} p(H \rho)\right|_{t_{i}} ^{t_{f}} ,
    \label{eq3.7}
\end{equation}
和
\begin{equation}
    W \equiv\langle w\rangle=\int_{t_{i}}^{t_{f}} \M{d} t \int \M{d} q \int \M{d} p \left[ \rho\left( \dot{\lambda} \frac{\partial U_0}{\partial \lambda} + \PP{U_1}{t} \right)\right] .
    \label{eq3.8}
\end{equation}

对于吸收的热的系综平均,由于存在$\dot{q}$和$\dot{p}$我们不好直接积分,需要做进一步的推导。系综平均为
\begin{equation}
    Q=\int_{t_i}^{t_f}\langle(\dot{x} \hat{\mathbf{x}}+\dot{p} \hat{\mathbf{p}}) \cdot \nabla H\rangle \mathrm{d} t
    \label{eq3.9}
\end{equation}
为了方便,上式\eqref{eq3.11}我们用了$(\dot{x} \hat{\mathbf{x}}+\dot{p} \hat{\mathbf{p}}) \cdot \nabla H  = \dot{p} p+\dot{q} \frac{\partial U}{\partial q} +  \dot{p} \PP{U}{p}$。
现在分两步对上式\eqref{eq3.9}进行平均。

第一步,对系综中在$t$时刻经过$x$且动量为$p$的轨道进行平均\R{(不懂至以下)},得到
\begin{equation}
    \langle\dot{x} \mid x, p, t\rangle=\frac{J_{x}}{\rho}, \quad\langle\dot{p} \mid x, p, t\rangle=\frac{J_{p}}{\rho}
    \label{eq3.10}
\end{equation}
其中$J_x$,$J_P$代表式\eqref{eq2.57}所定义的流矢量$J$的$q$分量和$p$分量。

第二步,利用分布函数$\rho(q,p,t)$对所有的$q$和$p$进行系综平均。于是,系统与热浴的热交换为
\begin{equation}
    Q=\int_{t_i}^{t_f} \mathrm{d} t \int \mathrm{d} q \int \mathrm{d} p(\bm{J} \cdot \nabla H)
    \label{eq3.11}
\end{equation}
再将流矢量$J$的定义代入上式\eqref{eq3.11},不难得到
\begin{equation}
    Q=-\int_{t_i}^{t_f} \mathrm{d} t \int \mathrm{d} q \int \mathrm{d} p\left[\gamma \rho\left(p+\frac{\partial U}{\partial p}\right)\left(p+\frac{\partial U}{\partial p}+\frac{1}{\beta \rho} \frac{\partial \rho}{\partial p}\right)\right]
    \label{eq3.12}
\end{equation}

\section{类卡诺循环热机模型}
\qquad 考虑一个由绝热捷径与等温捷径构成的类卡诺循环热机,它通过由一个依赖于时间的谐振子势$U_{0}(q, \lambda(t))= q^{2}/2{\lambda(t)}^2 $,来驱动布朗粒子对外做功。如图\ref{p3.1}所示,这个循环由如下四个过程组成。
\begin{figure}[!htbp]
    \centering
    \def\svgwidth{0.6\columnwidth}
    \input{figures/p3.1.pdf_tex}
    \caption{热力学循环过程。点线对应于式\eqref{eq2.40},竖直虚线对应于等温。$0,1,2,3$代表循环中的四个状态,$A,B,C,D$代表四个过程。}
    \label{p3.1}
\end{figure}
下面我们分过阻尼布朗粒子和欠阻尼布朗粒子两种情况进行讨论

\subsection{过阻尼布朗粒子构成的类卡诺循环热机}
\qquad 为了下面讨论的方便,先利用式\eqref{eq31.7},\ \eqref{eq31.8},\ \eqref{eq31.11}计算能量差$\Delta E$、输入功$W$、吸收热$Q$的方法。对于过阻尼布朗粒子的等温捷径,注意到根据式\eqref{eq2.55},$U=U_0 + U_1 = q^2/{2 \lambda(t)^2}  -{\gamma \dot{\lambda}(t)  q^{2}}/{2 \lambda(t)}$,又根据\eqref{eq2.46}的假设,$\rho(q, t)= \mathrm{e}^{\beta\left[F(\lambda(t))-q^2/{2 \lambda(t)^2}\right]},\ F(\lambda) \equiv-\beta^{-1} \ln \left[\int_{-\infty}^{+\infty} \mathrm{e}^{-\beta q^2/{2 \lambda(t)^2}} \M{d} q \right]$。于是不难得到,
能量差
\begin{equation}
    \begin{split}
        \Delta E &=\left. \frac{1}{2} \beta^{-1} (1-\gamma \dot{\lambda} \lambda) \right|_{t_i}^{t_f}\\
                 &=0 
    \end{split}
    \label{eq3.13}
\end{equation}
其中用到了对$\lambda(t)$的限制条件\eqref{eq2.51}。容易发现,和经典中理想气体的等温过程相似,粒子的能量没有变化。接下来计算输入功

\begin{equation}
    \label{eq3.14}
\end{equation}

\begin{center}
    {\bfseries A.等温膨胀}
\end{center}

在图\ref{p3.1}中,由实线连接的$0 \to 1$代表了由等温捷径实现的“等温膨胀”过程。正如在第\ref{cha2}章第\ref{sec2.3.2}节所阐释的,参数$\lambda$表征了系统的空间尺度,而该过程中参数$\lambda$变大了,故把过程A称为等温膨胀过程。

\section{类卡诺循环热机的功、熵、能量损失}

\section{与其他热机的比较}